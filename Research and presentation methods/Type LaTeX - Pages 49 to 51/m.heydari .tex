\documentclass[a4paper,12pt]{article} 
\usepackage{xepersian}
\usepackage{graphics}
\settextfont{XBZar}
\setdigitfont{XBZar}
\fontsize{14}{15}\selectfont




\begin{document}



\noindent
   نام : مهسا 
 نام خانوادگی : حیدری  \\
 شماره دانشجویی : 949728209 \\
 دانشجوی رشته مهندسی کامپیوتر (گرایش نرم افزار)\\
 درس:  روش پژوهش و ارائه \\
 استاد  : جناب  آقای دکتر رضوی  \\                                                                                                                                                                                                                                                                                                                                                                                
برنامه نویسی لاتک از صحه 49 تا 51 کتاب اصلی   \\  

   \noindent
این متن و برنامه نویسی  و حروف چینی آن   توسط نرم افزارلاتک  انجام گرفته است .                                     
\vspace{0.1cm}
\vspace{0.1cm}
\vspace{0.1cm}
\begin{center}

THE LITERATURE REVIEW PROCESS IN E-RESEARCH

\end{center}
      

\begin{latin}
\noindent
groups'  FAOs  are  linked  through the Internet  FAQ consortium at\\ http://www.faqs.org/.
 These lists  are designed to inform new  users about fre­ quently asked questions and  to  help   prevent regular readers from rereading responses to questions that  have  been  dealt  with  on  many  previous occasions. Use quotes to reference only  relevant material  from a previous post to which  you are  responding. A brief quotation is  useful  to provide context; however, long inclusions of past comments only  waste  bandwidth and  add  to screen clutter. Follow the  discourse  for  a few  weeks before  posting comments or  questions
yourself to  insure that  your particular question is relevant to the  interests of the list  members.

\vspace{0.1cm}
\vspace{0.1cm}
\vspace{0.1cm}
\noindent
If in doubt about the appropriateness of a potential  posting, email it privately to the list  owner for  feedback before posting it publicly.
The use of HTML coding and  the  addition of attachments to postings always add to  the  size of the  message and may  result in messages that cannot be read  b
all  members of the  list. A better solution  is to post  an announcement of the  avail­ ability  of the  resource to the  list, Usenet group, or virtual conference and to post longer or multimedia messages and  files  to a Web site where the   interested reader can selectively  retrieve them.


\vspace{0.1cm}
\vspace{0.1cm}
\vspace{0.1cm}
\noindent
Create a separate file folder in your mailbox for information you are  sent  when first subscribing to a new  list.  These  first subscriber information postings will tell you how  to resign  or suspend your membership in the  list-information that may be relevant but  very difficult to find  when you wish  to resign from the group or change your email address.


Virtual  Conferences




\noindent
The first  virtual conference" on  the  Internet was organized in  1992 fr the  Interna­ tional Council for  Distance Education (Anderson and Mason,  1993). Subsequently, vir­ tual conferences have proliferated and  continue to provide a forum   for  professional development that is much more cost-effective and  accessible than   their face-to-face equivalent (for example see  http://www.rmrple.com). Virtual conferences are   time­
delimited in that  they  use combinations of synchronous and asynchronous tools to sup­ port presentation and dialogue for a limited period of time  and  usually  on a particular topic (Anderson, 1996), Like  their face-to-face counterparts, virtual conferences usu­ ally include keynote presentations, promotional displays, and  small  group discussions. Such  conferences provide ideal  means  for  e-researchers to gain  low-cost exposure to major spokespersons in their field.  Announcements of upcoming virtual conferences can be found on appropriate mailing lists,  Usenet groups, or the  home pages  of spon­ soring organizations.


\vspace{0.1cm}
\vspace{0.1cm}
\vspace{0.1cm}

Direct Email

\noindent
Writing directly to an  expert in  the  field  may  be a useful  way  for  the e-researcher to gain  invaluable access  to the  "informal network." The  use of powerful search engines usually  allows one to enter the  expert's  name (in  quotations and   possibly with   a key word  appended, if the  name  is very common) and  find  a Web site  with relevant information, including the expert's email address and phone number. However, the novice e-researcher should be careful not to abuse this availability and should  utilize such contacts only when other, less  demanding forms of communication  have  been exhausted.Most experts whom you would like to reference in your literature review are very busy people-if they  were not, you probably wouldn't be interested in their work. In addition, most experts write books, publish articles, and create Web sites so that you can  gain  access to their ideas and comments. Attempts by the novice e-researcher to short circuit the  process and  go directly and  personally to the  expert,  without check­ ing  their public work, will  likely  be interpreted as bothersome and not be answered.
\vspace{0.1cm}
\vspace{0.1cm}
\vspace{0.1cm}


As an example of an inappropriate request, we recently received an email  froma
\noindent
graduate student studying in  a foreign country. The email noted that the student had read one of our articles, liked it,  and  wondered if we  could send more  information. Since we had no idea which article was  read (we have been  publishing for many years), we  were   not  inclined to  even   answer the  letter.  Alternatively,  legitimate,  well­ informed, clearly written, and polite questions and concerns may  not only be answered, but may be appreciated and lead to further contacts with experts in the field.

\vspace{0.1cm}
\vspace{0.1cm}
\vspace{0.1cm}
   

           

Filtering Messages for Others
\noindent

\vspace{0.1cm}
\vspace{0.1cm}
\vspace{0.1cm}
\noindent
It  is  impossible to follow all  of the discussion groups and  Web  sites that may have information relevant to your field of study. Thus,  many successful  e-researchers develop informal networks of friends and colleagues who filter relevant information from   their own explorations of the network and forward appropriate messages, links, or referrals to them. 'This filtering can become institutionalized as the  researcher sets up a formal or informal mailing list for messages or references that contain informa­ tion relevant to the members  of the list.  In the early days of networking, prior to the Internet, this transporting of information  between networks was  referred to as porting and porters were celebrated as  "Unsung heroes of the  Network Nation!" (Masthead, NerweaverMagazine).

\vspace{0.1cm}
\vspace{0.1cm}
\vspace{0.1cm}





Making Effective  Use of the Informal Network Resources


\noindent
To maximize the  effectiveness of an  inquiry, an  e-researcher must be careful to ask  a question or request assistance in  an  appropriate manner. s in any conversation, the
researcher must  be sure to use a manner and  tone that  is polite, respectful, and appre­ ciative.  In addition, e-researchers must insure that they  have done their own  literature review  and research work before asking others to do it  for  them.


\indent
For example, a question such as  "Does anyone know  anything about school 
\noindent
dropout for a research paper I'm doing?" will likely  not result in any assistance and will certainly let the members of the  group know you have a great deal  to learn about both the   subject and   the  etiquette of  Net-based discussions. A refined request such as Tito's model of student dropout seems to be used  often in  postsecondary, but an ERIC search turns up only a single study in a secondary school context. Does anyone on  this list  (Usenet group, or virtual conference) know  of any  work,   using Tinto'g  model  in this  area or have  any  ideas why it  is not appropriate?" This latter phrasing illustrates that you have done some  research and  thinking  and may well be a useful con­ tact and serious e-researcher.

\vspace{0.1cm}
\vspace{0.1cm}
\vspace{0.1cm}

\indent


Citing Net-based Resources in the Literature Review



\noindent
There are   a  number of  formats  for  referencing documents  and  correspondence obtained from the  Internet.  In general   the  format for  most  styles  follows  that  pre­ scribed for the  referencing of paper-based documents, with  the  addition of the  Uni­ form Resource Locator (URL) and  the date  of access of the  document appended to the end of the reference. For example, in American  Psychology Association  (APA) format the equivalent  paper  reference is followed  by the  words:
\vspace{0.1cm}
\vspace{0.1cm}
\vspace{0.1cm}
\indent

Retrieved  on date  from the  World  Wide  Web. bntp://site address

\vspace{0.1cm}
\vspace{0.1cm}
\vspace{0.1cm}



\noindent
Some citing  guidelines (notably APA) do not encourage private emails,  unarchived  list postings, or postings to Usenet groups  in the  reference bibliography, because obtain­ ing  a copy  of the  correspondence may  be difficult  or impossible for  the  interested reader. Instead, these guidelines suggest  referencing such  private or difficult to retrieve material  as "private  email correspondence from name on  date" or "posting to list  name on date" within  the  text of the document. Other guides  suggest that  this information be kept and made  available  to the interested reader and  that  it be referenced in the  bibli­ ography in the  format:
\vspace{0.1cm}
\vspace{0.1cm}
\vspace{0.1cm}

\indent
Anderson, T. (16 September  200D). Subject: When will our book be published? [email
to H.  Kanulea],  [Online]. Available email:\\  heather.knulaa@ualberta.ca

\vspace{0.1cm}
\vspace{0.1cm}
\vspace{0.1cm}

For   more  information related   to the   format  for  citing  electronic references,  the World-Wide Web  Virtual  Library  maintains a listing  of sites entitled Electronic Refer­ ences dScholarly Citations ofInternet Sources at http://www.spaceless.com/WWW/VL/ lt is important to learn  and consistently use the  format  in which your e-research
results will eventually  be published. Making  use of consistent notation of all relevant fields fromthe very beginning of the research  process  will save you  a great deal of tire in the long run. To aid  in this  data  organization process, a brief discussion of personal reference management software appears  in the  next  section,
\vspace{0.1cm}
\vspace{0.1cm}
\vspace{0.1cm}



\indent
PLAGIARISM  AND NETWORKED SOURCES
\noindent
Most academic writing has liberal doses of direct quotations from the  works of others This practice lends  authenticity to the  literature review and,  done properly, can even enhance the  readability of the  literature review.  However, it  is imperative that   the work  of others be properly acknowledged. Even if a quotation is not  used  directly, ideas that  are paraphrased by the  researcher need to be credited to the  original source. Given  the  pervasiveness of ideas, papers,  reference sources,  and  commentaries on  the Net, e-researchers may  feel  they are  drowning in a bewildering and immense sea of information, sources, and  references. They may even  have trouble remembering where  or even if the  idea  or quotation they   have  gathered came from  another source or is original work.  There is no easy  solution to this problem, except  to remember that qual­ ity  research is  systematic. An  electronic or  paper notebook (see  personal reference management software in the next  section for an  example) to record quotes  and ideas, as well  as  their source, is  an  essential  tool for  all  researchers and can  be especially Important when beginning a literature review



\end{latin}


\end{document}


